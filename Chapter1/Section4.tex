\section{Measurements}

During the first section we have introduced the concept of measure as a set $\{ \hat{\Pi}_n \}_{n\in\mathcal{I}}$ of operators that allow us to evaluate the probability of having a certain outcome $\lambda_n$ for the measure of a physical quantity $\Lambda$. Nevertheless, even if we have listed the main properties that such operators should have we didn't specify the forms that they can possess. It is now time to introduce the main way in which such set of operators can present themselves and how they will work inside quantum circuits.

\subsection{Projection valued measurement}

If we take a look back to the different properties stated in Def. (\ref{def:measure}) an idea should arise, especially by looking at the collapse condition. The latter, in fact, can be thought as the state of the system gets projected into a certain state that posses a definite value of the physical quantity we are interested in measuring. In fact, it's easy to see how the following is true
\begin{align}
    &\hat{\Pi}_n \ket{\psi_n} = \ket{\psi_n}, &p_n = \ev{\hat{\Pi}_n^\dagger\hat{\Pi}_n}{\psi_n} = 1.
\end{align}
Operators that posses this effect of taking a general state $\ket{\psi}$ and taking it into another are really known in mathematics and are called \textbf{projector operators}, represented as $\hat{P}$. This type of operator is defined, in particular, by two main conditions
\begin{align}
    \label{eq:ProjDef}
    &\hat{P}^\dagger = \hat{P}, & \hat{P}^2 = \hat{P},
\end{align}
these two properties are enough to give $\hat{P}$ a lot of power allowing it to define alone a subspace of the vector space in which is working. In our case this means
\begin{equation}
    \mathcal{P} = \left\{ \left.\ket{\psi}\right| \exists \ket{\phi} \in \mathcal{H}: \ket{\psi} = \hat{P}\ket{\phi} \right\},
\end{equation}
basically we are aiming to create a set $\{ \hat{P}_n \}$ so that $\mathcal{P}_n$ is the eigenspace of $\lambda_n$.

To find out the right projectors to create the complete set we can simply use the properties of the observable in QM. An observable quantity is represented in QM using an operator $\hat{\Lambda}$ and the possible values that $\Lambda$ can take, the outcomes, are the eigenvalues of that operator $\lambda_n$. Nevertheless, for a general operator $\lambda_n$ can be complex, or worst they can not exist, that is a problem since is absurd to mesure a complex number in an experiment, therefore we will make a further assumption giving out the following definition.
\dfn{Observables}
{
    An observable $\Lambda$ in QM is represented by a hermitian operator $\hat{\Lambda}$, so that
    \begin{equation}
        \hat{\Lambda}\dagger = \hat{\Lambda}.
    \end{equation}
}
\noindent
A really important theorem of linear algebra called \textbf{spectral theorem} allow us to say that, with this further assumption of hermiticy, the operator can be diagonalyzed and the eigenvalues are all real. Along with that, the spectral theorem also allow us to say that a set of eigenstates $\{\ket{\psi_n}\}$ with the following properties exist
\begin{equation}
    \label{eq:BaseDef}
    \hat{\Lambda}\ket{\psi_n} = \lambda_n\ket{\psi_n}, \hspace{2cm} \braket{\psi_n}{\psi_m} = \delta_{nm}, \hspace{2cm} \sum_n \ketbra{\psi_n}{\psi_n} = \mathbb{1},
\end{equation}
forming an orthonormal base for $\mathcal{H}$. These properties should hint us that the set of projectors that we want to create to measure $\Lambda$ may be the ones that project onto the eigenspace generated by $\ket{\psi_n}$, and we can easily see how that is the case. 

\thm{PVM measure}
{
    Taken $\hat{\Lambda}$ an observable the set of projectors $\{ \hat{P}_n \}_{n\in\mathcal{I}}$ onto the orthonormal base of eigenstate $\{\ket{\psi_n}\}_{n\in\mathcal{I}}$ of $\hat{\Lambda}$ forms a measure for the observable itself called \textbf{Projection valued measurement}.
}
\pf{Proof}
{
    We are going to define the projector $\hat{P}_n$ as follows
    \begin{equation}
        \hat{P}_n = \ketbra{\psi_n},
    \end{equation}
    which can be easily seen it's a projector, the demonstration is left to the reader. We want to see how all the requirement in Def. (\ref{def:measure}) are verified, we can start from the probability by taking a general state $\ket{\psi}$ and writing
    \begin{align}
        &\ket{\psi} = \sum_n c_n\ket{\psi_n},   & p_n = \ev{\hat{P}^\dagger_n\hat{P}_n}{\psi} = \ev{\hat{P}_n}{\psi} = \abs{c_n}^2.
    \end{align}
    Where I have first written the expansion of $\ket{\psi}$ on the orthonormal base and then used the properties in \eqref{eq:ProjDef} and \eqref{eq:BaseDef} to obtain the probability. It's possible to see how the values of $\abs{c_n}^2$ effectively represents probabilities since also the condition \eqref{eq:MeasureCompleteCondition} is respected by the set of operators defined thanks to the completeness condition of the orthonormal base
    \begin{equation}
        \sum_n \hat{P}^\dagger_n\hat{P}_n = \sum_n \hat{P}_n = \sum_n \ketbra{\psi_n} = \mathbb{1}.
    \end{equation}
    Therefore, the first condition for having a measure is respected. We can now see how also the second one is obtained as we want since we can write
    \begin{equation}
        \ket{\psi_n} = \frac{\hat{P}\ket{\psi}}{\sqrt{\ev{\hat{P}^\dagger_n\hat{P}_n}{\psi}}} = \frac{c_n}{\sqrt{\abs{c_n}^2}}\ket{\psi_n} = e^{i\theta}\ket{\psi_n},
    \end{equation}
    and since a complex phase doesn't change the physical state of the system the wanted result is achieved.
}

\ex{Qubit PVM}
{
    To make an example we can use a qubit, where we can imagine measuring the state as $\ket{1}$ or $\ket{0}$, so a computational base measurament. We can also easily imagine what is the form of the operator, the observable, that has them as eigenstates
    \begin{align}
        &Z = \begin{pmatrix}
            1 & 0\\
            0 & -1
        \end{pmatrix}, &Z\ket{i} = \lambda_i\ket{i},
    \end{align}
    where it's easy to see, by using the $\mathbb{C}^2$ representation of the states, how $\lambda_0 = 1$ and $\lambda_1 = -1$. This means that for every state $\ket{\psi}$ we can measure the probability of a certain outcome, state $\ket{1}$ or $\ket{0}$, to appear by using the set of operators defined by $\hat{P}_i = \ketbra{i}$ given as matrix by
    \begin{align}
        &P_0 = \begin{pmatrix}
            1 & 0\\
            0 & 0
        \end{pmatrix},
        &P_1 = \begin{pmatrix}
            0 & 0\\
            0 & 1
        \end{pmatrix}.
    \end{align}
    Applied to a general vector they will give in general the first or second component, respectively.
}

\subsection{Positive operator valued measurement}

In the PVM description of measure that we have given just now we use projector operators to predict the probabilities, which posses a series of properties that makes them really well suited for the task. Nevertheless, they are not the only possible choice, in particular such projectors posses the property of being hermitian that general measures can totally not have. We want so to see a case where another type of measure respect to the PVM can be a better choice to study the system.

Let's imagine having a qubit and a quantum circuit that is able to prepare it in two states given by
\begin{align}
    &\ket{\psi_1} = \ket{0}, & \ket{\psi_2} = \frac{\ket{0} + \ket{1}}{\sqrt{2}}.
\end{align}
We want to see if we are able to understand which state after a measure. Using the PVM measure of a qubit using $Z$ as an observable we can easily see how $\ket{\psi_1}$ posses a probability equal to $1$ of being in state $\ket{0}$, while $\ket{\psi_2}$ has $p_i = 1/2$ for both states. This means that if I take a measure and the outcome is $\lambda = -1$ I know that the only state that can have that outcome is $\ket{\psi_2}$ since has a non-zero probability of being in state $\ket{1}$. Instead, if the result is $\lambda = 1$ I can't say which one of the two state is the right one. We can also understand why we are not able to decide, since the two states that we are working with are not orthogonal respect to the computational base and so $\ket{\psi_2}$ has non-zero probability of being in both states. Therefore, we would like to use a set of operators that instead are able to give us this property, and a possible way of obtaining them is by taking $\{ \hat{\Pi}_n \}$ so that
\begin{equation}
    \hat{\Pi}_n^\dagger\hat{\Pi}_n = \hat{E}_n,
\end{equation}
are \textbf{positive valued operators}.