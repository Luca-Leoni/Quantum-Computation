\section{Entanglement}

The concept of entanglement comes from the need of describing the difference of certain type of states respect to others. In particular, we can understand it by first defining a really simple type of state called separable that has the following form.
\dfn{Separable state}
{
    Taken a state $\ket{\psi}\in\mathcal{H}_1\otimes\mathcal{H}_2$ it's called separable if $\exists \ket{\phi_1}\in \mathcal{H}_1, \ket{\phi_2} \in \mathcal{H}_2$ so that we can write
    \begin{equation}
        \ket{\psi} = \ket{\phi_1}\otimes\ket{\phi_2}.
    \end{equation}
}
\noindent
This type of states are really simple since we can decompose them into simpler once and work with them. To make an example we can see how the state $(\ket{01} + \ket{00})/\sqrt{2}$ is a separable one since
\begin{equation}
    \frac{\ket{01} + \ket{00}}{\sqrt{2}} = \ket{0}\left( \frac{\ket{0} + \ket{1}}{\sqrt{2}} \right),
\end{equation}
holds true and so can be decomposed. Nevertheless, we shall not go much further to find out more complex states that cannot be decomposed for which we need to work in the higher dimensional space like the Bell's state $(\ket{01} + \ket{10})/\sqrt{2}$. Those are entangled states, whose definition is therefore really simple as.
\dfn{Entangled states}
{
    A state $\ket{\psi}\in\mathcal{H}_1\otimes\mathcal{H}_2$ it's called entangled if it's not separable.
}
\noindent
Therefore, the mathematical definition of entanglement seems really simple, but the physical consequences are not, and we shall see together why this is one of the main powers of QC.