\section{Entanglement}

The concept of entanglement comes from the need of describing the difference of certain type of states respect to others. In particular, we can understand it by first defining a really simple type of state called separable that has the following form.
\dfn{Separable state}
{
    Taken a state $\ket{\psi}\in\mathcal{H}_1\otimes\mathcal{H}_2$ it's called separable if $\exists \ket{\phi_1}\in \mathcal{H}_1, \ket{\phi_2} \in \mathcal{H}_2$ so that we can write
    \begin{equation}
        \ket{\psi} = \ket{\phi_1}\otimes\ket{\phi_2}.
    \end{equation}
}
\noindent
This type of states are really simple since we can decompose them into simpler once and work with them. To make an example we can see how the state $(\ket{01} + \ket{00})/\sqrt{2}$ is a separable one since
\begin{equation}
    \frac{\ket{01} + \ket{00}}{\sqrt{2}} = \ket{0}\left( \frac{\ket{0} + \ket{1}}{\sqrt{2}} \right),
\end{equation}
holds true and so can be decomposed. Nevertheless, we shall not go much further to find out more complex states that cannot be decomposed for which we need to work in the higher dimensional space like the Bell's state $(\ket{01} + \ket{10})/\sqrt{2}$. Those are entangled states, whose definition is therefore really simple as.
\dfn{Entangled states}
{
    \label{def:Entangl}
    A state $\ket{\psi}\in\mathcal{H}_1\otimes\mathcal{H}_2$ it's called entangled if it's not separable.
}
\noindent
Therefore, the mathematical definition of entanglement seems really simple, but the physical consequences are not, and we shall see together why this is one of the main powers of QC.

\ex{}
{
    To make an example of how powerful the entanglement can be we can see how in theory we can transport the information of two bits, so $00$, $01$, $10$ or $11$, transmitting only one qubit. The idea is that two people, Alice and Bob, posses two qubit that are entangled in the following Bell state
    \begin{equation}
        \ket{\psi^+} = \frac{\ket{00} + \ket{11}}{\sqrt{2}},
    \end{equation}
    then Alice can perform one of four operations on it's qubit based on what wants to transmit to Bob. In particular the following choices can be made
    \begin{align*}
        &\text{Operation}               &\text{Final state}\\
        &\ket{00} \to \mathbb{1}        &\ket{\psi^+}\hspace{0.4cm}\\
        &\ket{01} \to Z                 &\ket{\psi^-}\hspace{0.4cm}\\
        &\ket{10} \to X                 &\ket{\phi^+}\hspace{0.4cm}\\
        &\ket{11} \to iY                &\ket{\phi^-}\hspace{0.4cm}
    \end{align*}
    So, based on the information that we want to send we have performed an operation of the Alice qubit to obtain a different Bell state. Then, Alice only need to send the qubit to Bob which need to perform a two qubit measure using the Bell's base and the state that will obtain will correspond to the selected two bit information obtained with the transport of only one qubit.
}

\nt
{
    This definition of Entanglement is not really the most general one that can be used. In fact, the Def. (\ref{def:Entangl}) it's a specific one called bi-entanglement since only two Hilbert spaces are counted, but the most general called n-entanglement count for $\ket{\psi}\in \bigotimes_{i=1}^n \mathcal{H}_i$. Nevertheless, for QC application only the bi-entanglement plays an important role.
}

\subsection{Quantum teleportation}

The first real important application of entanglement that we will see it's the quantum protocol for the quantum teleportation of information. Basically we want to teleport a quantum state from one qubit to another, so that no matter is transported to one place to another but only the state of the qubits is changed transporting information. Also, at first site one can also think that the cloning theorem would be violated in a situation of this type, but that is not the case, and we will see why.
\begin{figure}[b]
    \centering
    \includegraphics[width=0.8\textwidth]{Immagini/QuanTeleport.pdf}
    \caption
    {
        Quantum teleportation circuit for a quantum computer, see how it involves the measuring of two qubit to control two operations.
    }
    \label{fig:QuanTeleport}
\end{figure}

The circuit that is used in order to perform the teleportation protocol is reported in \figref{fig:QuanTeleport}, and consist in the use of two qubit in order to transport a general state $\ket{\psi} = a\ket{0} + b\ket{1}$ into one of those qubit. Thus, we want to understand how this is possible in general, and to do that let's pretend that the two qubit $\ket{\psi}$ and $Q_1$ are in possession of Alice, while $Q_2$ is with Bob. We will need that the two qubit $Q_1$, $Q_2$ are entangled in a $\ket{\psi^+}$ Bell state, then we can place them at whatever distance, so that Alice and Bob can be also on different planets. In this situation Alice will perform a series of operation on her qubit starting with a CNOT, to understand what this operation will do on the system we need first to look at the state of the hole three qubit, which will be
\begin{equation}
    \ket{\psi Q_1Q_2} = \left( a\ket{0} + b\ket{1} \right)\left( \frac{\ket{00} + \ket{11}}{\sqrt{2}} \right) = \frac{1}{\sqrt{2}}\left[ a\ket{000} + a\ket{011} + b\ket{100} + b\ket{111} \right].
\end{equation}
Then we can easily perform the CNOT on the first two qubit ending up in the following state
\begin{equation}
    \frac{1}{\sqrt{2}}\left[ a\ket{000} + a\ket{011} + b\ket{110} + b\ket{101} \right],
\end{equation}
on which the Hadamart tranformation is applied on the first qubit, which we shall remember that transforms $\ket{0}$ in $\ket{+}$ and $\ket{1}$ in $\ket{-}$. Therefore, we will have the following form
\begin{equation}
    \frac{1}{2}\left[ a\ket{000} + a\ket{100} + a\ket{011} + a\ket{111} + b\ket{010} - b\ket{110} + b\ket{001} - b\ket{101} \right],
\end{equation}
which can be rewritten by seeing how inside it four separable states can be found out, having
\begin{equation}
    \frac{1}{2}\left[ \ket{00}\left( a\ket{0} + b\ket{1} \right) + \ket{01}\left( a\ket{1} + b\ket{0} \right) + \ket{10}\left( a\ket{0} - b\ket{1} \right) + \ket{11}\left( a\ket{1} - b\ket{0} \right) \right].
\end{equation}
At first sight this state may seem nothing special, but if the states inside parentheses are inspected one could find out different forms of $\ket{\psi}$ obtaining
\begin{equation}
    \frac{1}{2}\left[ \ket{00}\left( \ket{\psi} \right) + \ket{01}\left( X\ket{\psi} \right) + \ket{10}\left( Z\ket{\psi} \right) + \ket{11}\left( XZ\ket{\psi} \right) \right].
\end{equation}
We have so created a state where the two qubit state given by Alice's couple is entangled to the qubit of Bob, which will poses the $\ket{\psi}$ state up to two transformation based on the Alice's values. Therefore, if Alice measure it's two qubit and refers the outcomes to Bob, he can perform the right operations to obtain at the end the final state $\ket{\psi}$ on his qubit. An operation that inside the circuit is described by the two classical controls and can be seen gives the right result keeping in mind that $X^2 = Z^2 = \mathbb{1}$.

Thus, using this protocol a general state $\ket{\psi}$ can be transferred into another qubit at whatever distance without limitations. In fact, the circuit has been performed in several occasions starting from a teleportation distance of some centimeters to the experiment of Anton Zellinger that performed it at nearly one kilometer of distance inside Vienna. Today we are able to transfer states from hearth to satellite, but still this phenomenon result in being quite strange at first sight. Perhaps, as was told earlier, one can say that this \textbf{phenomenon contradicts Thm. (\ref{thm:cloning})} with the coping of a general state into another one. This is a wrong affirmation, since the non-cloning theorem tells that the following operation doesn't exist
\begin{equation}
    \mathcal{U}(\ket{\psi}\otimes\ket{s}) = \ket{\psi}\otimes\ket{\psi},
\end{equation}
while here the starting states have been measured by Alice, meaning that they have collapsed having a result more similar to
\begin{equation}
    \mathcal{U}(\ket{\psi}\otimes\ket{Q_1}\otimes\ket{Q_2}) = \ket{00}\otimes\ket{\psi}.
\end{equation}
Where $\ket{00}$ is only one of the four possible final result obtained. Another apparent contradiction that can be found out inside the teleportation protocol is the fact that \textbf{information seem to pass from one qubit to another faster than the speed of light}. That can't obviously be possible, since relativity forbid it, and in fact that is not the case. The two classical controls that are used inside the circuit solve the problem since in order to obtain the copied state inside $Q_2$ I first need Alice to transmit the result of the measurements to Bob and that classical exchange of information take times.

\subsection{EPR paradox and Bell's inequality}

Since in 2022, the year I'm writing these notes, the Nobel Prize in physics was awarded to Alain Specter, John Clauser and Anton Zeilinger for their experiments on entangled photons, a little part of the course was devoted to the description of the theory they allowed to uncover. To understand it we shall understand that in the first half of 1900 the idea of entanglement was difficult to accept. Especially Einstein was really skeptical about it since at first sight looked like a threat to its restricted relativity with information apparently traveling faster than light. Remained in history Einstein's definition of entanglement as "spooky action at a distance", showing how much he hated it. In this context, great work was putted by Einstein and others physicist in order to see if effectively QM was wrong or, better, not entirely right, meaning that we were still missing something to understand it at its fullest.

In 1935 Einstein, Podolsky and Rosen (EPR) delineated a theory trying to show that QM was indeed not complete, starting by defining what it means for a physical theory to be complete. The line of reasoning starts with the definition of the so-called elements of reality.
\dfn{Elements of reality}
{
    A quantity is called element of reality if it has a value that can be predicted before experiment, basically it's already perfectly known already before taking the measure. A property of this type, for logic, must be owned by the system regardless of what you are making so that exist also before taking the measurement with a precise value.
}
\noindent
This definition is purely based on logic, assuming that if I'm able to know the exact outcome of an experiment before taking the measure then the property that I'm measuring needs to have that value defined intrinsically inside it. Then, the next move is to use this definition to define when a theory is complete.
\dfn{Complete theory}
{
    A physical theory is complete if it contains all elements of reality.
}
\noindent
Using these two definitions, the three theoreticians demonstrated that QM is not complete if we assume the notion of locality. Where, locality was also defined by them as follows.
\dfn{Locality}
{
    Measurements are independent if done in position and time that are not causally related.
}
\noindent
With this they were able to create a thought experiment that demonstrated how some elements of reality were missing inside QM. The experiment that E. P. R. proposed is actually old, today the best way to understand this is by looking at the one proposed by David Bohm in 1951 which shows how spin is what is missing.
\thm{Incompleteness of QM}
{
    Spin is an element of reality that is not contained entirely inside quantum mechanics.
}
\pf{Proof}
{
    We can imagine taking a singlet state of spin which is given by the following Bell state
    \begin{equation}
        \ket{\phi^-} = \frac{\ket{\uparrow\downarrow} - \ket{\downarrow\uparrow}}{\sqrt{2}}.
    \end{equation}
    This state is a peculiar one, since it's possible to demonstrate that has the same form in every possible base. Meaning that if I change the base of representation to an orthonormal couple $\{ \ket{e_1}, \ket{e_2} \}$ the state become
    \begin{equation}
        \ket{\phi^-} = \frac{\ket{e_1e_2} - \ket{e_2e_1}}{\sqrt{2}}.
    \end{equation}
    Thus, if one measures the spin of the system in any direction, so using the right base, the same result will be found out meaning that the spin is an element of reality with a defined value. Nevertheless, in QM we know that the spin operators do not commute, meaning that we can't know both components of the spin at the same time. Basically, even if the whole spin vector is an element of reality quantum mechanics only allows for the partial knowledge of it, meaning that QM does not contain all elements of reality.
}
\noindent
This result showed, in the minds of Einstein and collegues, that the QM is not a complete theory needing still work to find out certain \textbf{hidden variables} whose treated statistically will recreate quanutm mechanics but used normaly will give deterministic results. Basically he was searching for the equivalent of classical mechanics starting from statistical mechanics, where now statistical mechanics is quantum mechanics itself.

This discussion was incredibly interesting at the time, but understanding what the hidden variables were remained more of a phylisofic question for a lot of times untill the idea of Bell. In 1964, he pusblished a paper showing how whatever deterministic real local theory using hidden variables is doomed to fail in reproducing the quantum mechanical result using the following reasoning. He focused his studies on a particualar observable defined using four different ones $Q, S, R$ and $T$, where all of them could only have two possible outcomes $\pm 1$. The observable that we are interested in is the following
\begin{equation}
    \mathcal{L} = QS + RS + RT - QT,
\end{equation}
which posses some interesting properties such as it can have only two possible outcomes $\pm 2$. Bell showed that inside a local real deterministic theory the following result must hold.
\thm{Bell's inequality}
{
    Assuming locality and reality of the four observable composing $\mathcal{L}$ we have that the following relation hold
    \begin{equation}
        \left\langle \mathcal{L} \right\rangle \le 2.
    \end{equation}
}
\pf{Proof}
{
    We will assume that the four observables will be evaluated separetedly in four different measurements, having so that every draw forms the outcome $(q, r, s, t)$ with a probability $P(q, r, s, t)$.  We can so see how the following relation is true since we will have
    \begin{equation}
        \left\langle \mathcal{L} \right\rangle = \sum_{qsrt}(qs + rs + rt - qt)P(q, r, s, t) \le 2\sum_{qrst} P(q, r, s, t),
    \end{equation}
    but the sum of the probability of all possible outcomes need to be normalized at $1$ having so that the wanted relation holds.
}

\noindent
This result allowed to set a condition for the existence of the local hidden variables, the only thing remained to do is to show that quantum mechanics do not respect it. We can so take the following example 
\begin{align}
    &Q=Z_1, &S=-\frac{Z_2 + X_2}{\sqrt{2}},\\
    &R=X_1, &T=\frac{Z_2 - X_2}{\sqrt{2}},
\end{align}
and evaluate the average of $\mathcal{L}$ in the singlet state. From locality, we can say that no correlation exist from the outcomes of the different measurements, so that their probability are independent of each other having
\begin{equation}
    P(q, r, s, t) = P(q)P(r)P(s)P(t).
\end{equation}
Meaning that the following is true
\begin{equation}
    \ev{\mathcal{L}}{\phi^-} = \left\langle QR \right\rangle + \left\langle RS \right\rangle + \left\langle RT \right\rangle - \left\langle QT \right\rangle = 2\sqrt{2} > 2.
\end{equation}
Meaning that QM doesn't respect Bell's inequality and so can't be reconstructed using local hidden variables. This line of reasoning was also proven experimentally by Cluaser first, and then also by Specter with a more refined experiment that reproduced accuratelly the $2\sqrt{2}$ result using photons and not spins. Therefore, local hidden variables will not be the answer to find out a deterministic quantum theory.

From that point on quantum information evolved a lot with the creation of quantum optics and entangled photons, in particualar the last part of the Nobel Prize was given to Zellinger sicne was the first to perform quantum teleportation in open air at a far distance in Vienna. He teleported a state using entangled photons from one part to the other of the Danube river.